In recent years the biomedical industry has shown increased interest in using numerical methods to assist in the R\&D of medical devices. The long term goal is to reduce the costly and lengthy process that clinical trials take to for the US Food and Drug Administration (FDA) to approve a medical device. For this reason both FDA and academia are working together to create laboratory experiment that will help the industry gain confidence in numerical techniques as well as provide software developers with insights on the deficiencies and strength that numerical software may have. In this article three benchmarks proposed by the FDA are used to compare experimental results with those of the Finite element method (FEM) and Enhanced Particle Finite Element Method (PFEM-2). The first benchmark problem is the flow in a nozzle
containing a gradual and sudden change of diameter with the goal of predicting hemolysis (not studied in this work). The second problem studies the flow in a simplified centrifugal
blood pump under various pump operation conditions. Finally the third benchmark studies the steady flow in a patient-averaged inferior vena cava. PFEM-2 is regarded as a tool with great potential mainly because no stabilization is needed for the Galerkin approximation of the advection term in the transport equations. This could be a big advantage in problems with flows at high Reynolds number. The improved integration along streamlines provide a more accurate way to analyze problem with large time step. This paper is an effort to test PFEM-2 in real world engineering applications.
